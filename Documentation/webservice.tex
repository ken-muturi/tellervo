\chapter{Webservice specifications}
\label{txt:webserviceSpec}

This chapter outlines the basic syntax required for talking to the Tellero webservice and is largely aimed at developers who want to create new clients to interact with the Tellervo server.

The Tellervo database is accessed solely through the webservice interface.  A webservice is an interface designed to be accessed by programs that send requests and receive responses.  Tellervo uses a style of HTTP+POX (Plain Old XML) to send and receive requests via an HTTP POST.  In simple terms the Tellervo client sends an XML document that describes the request via POST to the Tellervo server.  The server then reads the XML request, performs the request and then compiles the information that has been requested, finally returning the information to the client as another XML document.  The syntax of the XML document containing the request and response is determined by the Tellervo XML schema and makes heavy use of the TRiDaS XML schema for describing dendrochronological entities.

The Tellervo schema is the final authority on what is allowed in a request or response so if you find a conflict between this documentation and the schema then it is most likely because the documentation is out-of-date, or it's simply incorrect.  As you become familiar with the schema you'll probably find it easier to refer to it rather than this documentation to understand what is expected.

  


\tip{One of the best ways of understanding the structure of requests and reponses is to use the XML monitor in the Tellervo Help menu to view the documents being sent and received by the Tellervo client. }

\section{Basics of sending requests}
The webservice accepts requests via the POST mechanism of HTTP.  The simplest way to understand this is to think of this as a standard webpage.  If you add the following HTML code to a webpage it will give you a simple form.  If you type your XML request document into the form and click submit the Tellervo server will respond with an XML document containing the answer to your query.

\begin{lstlisting}
<form method="post" action="http://name.of.your.server/tellervo/">
  <textarea name="xmlrequest"/>
  <input type="submit" value="submit" >
</form>
\end{lstlisting}

A page with just such a form is available in the root of your Tellervo server installation as the page post.php.

When the response is viewed in a web browser, it will be rendered using a style sheet, but if you view the source code, you'll see the underlying XML document.  A simple HTML form like this can be handy for helping you send some rough requests to the server to help you understand how things work.

The Tellervo client (or any other client that wants to talk to the Tellervo server) does exactly this.  It sends XML formatted request documents to the server as POST requests, and the server returns an XML document containing the reply.  The client then reads the XML reply and displays the result to the user in any way it sees fit.


\section{Standard request/response}

The request XML document you send the server needs to validate against the Tellervo XML schema (available in the schema folder of your Tellervo server or in the /src/main/resources/schemas/ folder in the Subversion repository.  The schema is a detailed representation of what tags are allowed, when they are obligatory and what possible values they can contain.  We strongly suggest getting hold of an XML validation tool to help you check that the requests you send the server are valid.  The server will do the same and respond with an error message if it is not valid, but the textual error is a lot harder to understand than the graphic display of a desktop validation tool.

The general layout of the request file is as follows:

\begin{lstlisting}
<?xml version="1.0" encoding="UTF-8"?>
<tellervo xmlns="http://www.tellervo.org/schema/1.0" xmlns:gml="http://www.opengis.net/gml" xmlns:xlink="http://www.w3.org/1999/xlink" xmlns:tridas="http://www.tridas.org/1.2.2">
  <request type="">
    ...
    ...
  </request>
</tellervo>
\end{lstlisting}


\begin{itemize}
 \item \textbf{Line 1} contains the XML declaration and tells the server that we're using UTF-8 character encoding.  This is the only encoding currently supported.
 \item \textbf{Line 2} starts the root tag and defines the namespaces used by Tellervo.  In this example the default namespace is the Tellervo schema itself, but it also refers to the TRiDaS, GML and XLink namespaces that Tellervo also makes use of. See section \ref{txt:Namespaces} for more details.
 \item \textbf{Line 3} begins the request tag which contains the request itself. 
\end{itemize}

When you send such a request XML document to the Tellervo server the typical response returned is structured as follows:

\begin{lstlisting}
<?xml version="1.0" encoding="UTF-8"?>
<tellervo xmlns="http://www.tellervo.org/schema/1.0" xmlns:gml="http://www.opengis.net/gml" xmlns:xlink="http://www.w3.org/1999/xlink" xmlns:tridas="http://www.tridas.org/1.2.2">
  <header>
    ...
  </header>
  <content>
    ...
  </content>
</tellervo>
\end{lstlisting}

The header contains standard information about the request and also includes error and warning messages.  The content tag contains the actual data being returned.

\subsection{Namespaces}
\index{Namespaces}
\index{XML!Namespaces}
\label{txt:Namespaces}

For newcomers to XML, namespaces and their definition can be quite confusing. Namespaces are used in XML to enable us to incorporate multiple schemas within the same XML file.  

In the case of Tellervo, that means we can use entities defined in both TRiDaS (to describe dendro data) as well as for instance GML (to describe location information).  Rather than reinvent the wheel, we use established standards like GML so that we can leave the experts in each field to handle their own datastands.  We need a method of clarifying which schema each entity in the XML file is referring to and for that we use namespaces.  

Namespaces are typically defined in the root tag of the XML file using the attribute `xmlns'.  Multiple namespaces can be described by adding prefix definitions.  For example \lstinline$xmlns:tridas='http://www.tridas.org/1.2.2'$ means that any tag prefixed by `tridas:' in the XML document refers to an entry in the TRiDaS schema.  Namespaces are URIs and are typically URLs.  There is quite often documentation about the schema at the namespace URL, but this is not necessarily the case.  Information is certainly not retrieved from this location.  The namespace just needs to be unique and must match what is defined by the schema.  The prefix itself though can be whatever you want.  If you have just a few prefixes, then it is common to just use a single character prefix, perhaps `t' for TRiDaS to keep the file small.  Some people prefer to be more explict though and using longer prefixes to ensure there are no confusions.  Table \ref{tbl:namespaces} lists the namespaces used by Tellervo.


\begin{table}
\label{tbl:namespaces}
\begin{center}
\begin{tabular}{ll}

\toprule
Schema & Namespace \\
\midrule
Tellervo & http://www.tellervo.org/schema/1.0\\
TRiDaS & http://www.tridas.org/1.2.2\\
GML & http://www.opengis.net/gml\\
XLink & http://www.w3.org/1999/xlink\\
\bottomrule
\end{tabular}
\caption{The namespaces used in the Tellervo schema}
\end{center}
\end{table}


If the majority of your XML file contains tags from one namespace, with just a handful of tags from another, you may prefer to define a default namespace.  In this case all tags from the default namespace do not require prefixes.  The following namespace declarations are therefore all valid and equivalent:

\begin{itemize}
 \item \lstinline$<tellervo xmlns="http://www.tellervo.org/schema/1.0" xmlns:gml="http://www.opengis.net/gml" $ \\ 
  \lstinline$xmlns:xlink="http://www.w3.org/1999/xlink" xmlns:tridas="http://www.tridas.org/1.2.2">$
 \item \lstinline$<t:tellervo xmlns:t="http://www.tellervo.org/schema/1.0" xmlns:gml="http://www.opengis.net/gml" $ \\   \lstinline$xmlns:xlink="http://www.w3.org/1999/xlink" xmlns:tridas="http://www.tridas.org/1.2.2">$
 \item \lstinline$<tellervo xmlns="http://www.tellervo.org/schema/1.0" xmlns:g="http://www.opengis.net/gml" $ \\   \lstinline$xmlns:x="http://www.w3.org/1999/xlink" xmlns:t="http://www.tridas.org/1.2.2">$
\end{itemize}


\subsection{Errors and warnings}
\label{txt:errorsandwarnings}
The webservice uses two tags within the header section of the return document to inform about errors and warnings.  These tags are the status and message tags.  The status tag can be set to: OK; Notice; Warning; or Error.  

When there is an error, the message tag includes a code attribute along with the actual error message.  The list of possible error codes is provided in appendix \ref{txt:errorcodes}.  An example of a typical error response is shown below:

\begin{lstlisting}
<?xml version="1.0" encoding="UTF-8"?>
<c:tellervo xmlns:c="http://www.tellervo.org/schema/1.0">
  <c:header>
    <c:securityUser id="1" username="admin" firstName="Admin" lastName="user" />
    <c:webserviceVersion>1.1.0</c:webserviceVersion>
    <c:clientVersion>Tellervo WSI null (httpcore 4.1; HttpClient 4.1.1; ts null.${revisionnum})</c:clientVersion>
    <c:requestDate>2012-04-02T15:01:59-04:00</c:requestDate>
    <c:queryTime unit="seconds">0.07</c:queryTime>
    <c:requestUrl>/tellervo/</c:requestUrl>
    <c:requestType>delete</c:requestType>
    <c:status>Error</c:status>
    <c:message code="907">Foreign key violation.  You must delete all entities associated with an object before deleting the object itself.</c:message>
  </c:header>
</c:tellervo>
\end{lstlisting}


\section{Authentication requests}
\index{Authentication}
There are two methods for authenticating yourself against the Tellervo server: plain and secure.  We \emph{strongly} recommend you use the secure method as the user name and password are sent in plain text over the internet when using the plain method.  This goes against so much of the hard work we've put in to making the system secure.  It is quite likely that we will disable the plain authentication method by default in the future.  For now it will be left in place as it makes testing new clients much easier.  

For further details and discussion about the authentication design please see section \ref{txt:authentication}, page \pageref{txt:authentication}.

\subsection{Plain authentication}
\index{Authentication!Plain}
If you \emph{still} want to go ahead and use plain authentication despite all the risks, then this is how you do it.

\begin{lstlisting}
<?xml version="1.0" encoding="UTF-8"?>
<tellervo xmlns="http://www.tellervo.org/schema/1.0" xmlns:tridas="http://www.tridas.org/1.3">
  <request type="plainlogin">
     <authenticate username="yourusername" password="yourpassword" />
  </request>
</tellervo>
\end{lstlisting}

\subsection{Secure authentication}
\index{Authentication!Secure}
Although this is much more secure, it is also somewhat more complicated because it involves a challenge and response scheme using cryptographic nonces and hashes.  

The first step is for the client to request a nonce from the server.  This is done as follows:

\begin{lstlisting}
<?xml version="1.0" encoding="UTF-8"?>
<c:tellervo xmlns:c="http://www.tellervo.org/schema/1.0" xmlns:gml="http://www.opengis.net/gml" xmlns:xlink="http://www.w3.org/1999/xlink" xmlns:tridas="http://www.tridas.org/1.2.2">
  <c:request type="nonce" />
</c:tellervo>
\end{lstlisting}

The returned document will include a nonce header tag like this: \lstinline$<c:nonce seq="176378">97566d4d2e8b8c5696b6667fef8429f5</c:nonce>$.  Armed with your nonce you can then send a request to log in securely as follows:

\begin{lstlisting}
<?xml version="1.0" encoding="UTF-8"?>
<c:tellervo xmlns:c="http://www.tellervo.org/schema/1.0" xmlns:gml="http://www.opengis.net/gml" xmlns:xlink="http://www.w3.org/1999/xlink" xmlns:tridas="http://www.tridas.org/1.2.2">
  <c:request type="securelogin">
    <c:authenticate username="admin" cnonce="3f975c569f978731e570" snonce="97566d4d2e8b8c5696b6667fef8429f5" hash="d315ec50f7f1809492d5ef132ad4aa06" seq="176378" />
  </c:request>
</c:tellervo>
\end{lstlisting}

The authenticate attributes are filled as follows.  The username is simply the username for a user with permission to access the Tellervo server.  The cnonce (client nonce) is a random string of your choosing that is used in the hash.  The snonce (server nonce) is the nonce you've just obtained from the server.  The seq (sequence) is the value also obtained from the server.  Finally the hash is an MD5 hash of ``username:md5hashofpassword:snonce:cnonce''.  If all is well the server will respond with the details of the person that is logging in.

\subsection{Cookies and sessions}
\index{Cookies}
The webservice uses a session cookie so that the user doesn't need to authenticate with each request.  The cookie lasts for up to 30 minutes of inactivity, after which point the server will request the user to re-authenticate before it will serve a request.  Any client attempting to access the Tellervo server will therefore need to handle cookies and be ready to respond to requests to re-authenticate.  

If a session does time out, a request to the server will result in an response with the header containing an error status with code `102' (see section \ref{txt:errorsandwarnings} and appendix \ref{txt:errorcodes}) and the nonce tag ready for the client to re-authenticate.  

\subsection{Logout}

The session cookie will ensure that a user is logged out after a period of inactivity, but if you want to force a logout you can simply use the following type of request:

\begin{lstlisting}
<?xml version="1.0" encoding="UTF-8"?>
<c:tellervo xmlns:c="http://www.tellervo.org/schema/1.0">
  <c:request type="logout">
</c:tellervo> 
\end{lstlisting}


\section{Reading records}

The method for reading records from the Tellervo database is largely the same for any of the types of data the server handles.  The basic template for a read request is as follows:

\begin{lstlisting}
<?xml version="1.0" encoding="UTF-8"?>
<c:tellervo xmlns:c="http://www.tellervo.org/schema/1.0">
  <c:request type="read" format=" ">
    <c:entity type=" " id=" " />
  </c:request>
</c:tellervo>
\end{lstlisting}

The entity type is one of: project; object; element; sample; radius; measurementSeries; derivedSeries; box; securityUser; or securityGroup.  The id attribute should be the database identifier for the entity you would like to read.  In Tellervo these are typically a UUID like this: 339d8ea6-7448-11e1-ad85-9b6d022add7a.  The format attribute should be one of: minimal; summary; standard; or comprehensive depending on how much detail you require about the entity.  Keep in mind that a comprehensive request is likely to take much longer to fulfill than a minimal request so it's best for your user, if you use the simplest request that fulfills your need.


\section{Deleting records}

The method for deleting records in Tellervo is very similar to reading records.  You simply use the request type `delete' and specify the entity type and id.  The basic template for a delete request looks like this:

\begin{lstlisting}
<?xml version="1.0" encoding="UTF-8"?>
<c:tellervo xmlns:c="http://www.tellervo.org/schema/1.0">
  <c:request type="delete">
    <c:entity type=" " id=" " />
  </c:request>
</c:tellervo>
\end{lstlisting}

Cascading deletes are not permitted in Tellervom therefore only entities that are not used by other entities in the database can be deleted.  For example, you cannot delete an object which has elements associated, you would need to delete in the elements first.  Attempts to delete an entity which still has associated records will cause the webservice to return a `907-Foreign key violation' error.



\section{Creating records}

New records are are typically created by passing a TRiDaS representation of the entity you'd like to create inside a `create' request.  The basic template of a create request looks like this:

\begin{lstlisting}
<?xml version="1.0" encoding="UTF-8"?>
<tellervo xmlns="http://www.tellervo.org/schema/1.0" xmlns:tridas="http://www.tridas.org/1.2.2">
  <request type="create" parentEntityID=" ">
    <tridas:sample>
      ...
    </tridas:sample>
  </request>
</tellervo>
\end{lstlisting}

The request tag should include the parentEntityID attribute with the ID of the parent entity in the database e.g.\ the TRiDaS element to which a sample belongs for instance.  Please note that there are a number of mandatory fields for each TRiDaS entity.  These must be populated otherwise the webservice will return a `902-Missing user parameter' error.

In addition to handling the standard TRiDaS project, object, element, sample, radius, measurementSeries, derivedSeries entities, the Tellervo webservice can also create some Tellervo specific records.  These are securityUser, securityGroup, box and permission.  How to create and alter permissions records is described in more detail in section \ref{txt:wsperms} as they are more complicated, but securityUser, securityGroup and boxes are handled in the same way as TRiDaS entities.


\section{Updating records}

Updating existing records in Tellervo is done in much the same way as creating records.  The basic template is as follows:

\begin{lstlisting}
<?xml version="1.0" encoding="UTF-8"?>
<c:tellervo xmlns:c="http://www.tellervo.org/schema/1.0" xmlns:tridas="http://www.tridas.org/1.2.2">
  <c:request type="update">
    <tridas:element>
      ...
    </tridas:element>
  </c:request>
</c:tellervo>
\end{lstlisting}

The main difference is that you \emph{must} specify the TRiDaS \lstinline$<identifier>$ tag containing the ID of the record you are trying to update. 


\section{Reading and setting permissions}
\label{txt:wsperms}
\index{Developing!Permissions}
\index{Permissions}



\begin{lstlisting}
<request type="create">
  <permission>
      <permissionToCreate>true</permissionToCreate>  
      <permissionToRead>true</permissionToRead>
      <permissionToUpdate>true</permissionToUpdate>
      <permissionToDelete>true</permissionToDelete>
      <entity type="object" id="760a19e2-229c-11e1-8756-03b2aff2fe33"/>
      <securityGroup id="3"/>
  </permission>
</request>
\end{lstlisting}


\begin{lstlisting}
<request type="read">
  <permission>
      <entity type="object" id="760a19e2-229c-11e1-8756-03b2aff2fe33"/>
      <securityGroup id="3"/>
  </permission>
</request>
\end{lstlisting}


\begin{lstlisting}
<request type="update">
  <permission>
      <permissionToCreate>false</permissionToCreate>  
      <permissionToRead>false</permissionToRead>
      <permissionToUpdate>false</permissionToUpdate>
      <permissionToDelete>false</permissionToDelete>
      <entity type="object" id="136a70a6-566b-546b-a3ae-c48cb046e4cd"/>
      <securityGroup id="1"/>
      <securityGroup id="3"/>
  </permission>
</request>
\end{lstlisting}


\begin{lstlisting}
<request type="delete">
  <permission>
      <entity type="object" id="136a70a6-566b-546b-a3ae-c48cb046e4cd"/>
      <securityGroup id="1"/>
  </permission>
</request>
\end{lstlisting}



