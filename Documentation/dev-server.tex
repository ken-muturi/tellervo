\chapter{Developing Corina Server}
\index{Developing|(}
\index{Developing!Webservice}
\index{Webservice!Developing}

The Corina server is made up of a PHP webservice run by Apache, connecting to a PostGreSQL database.  

The Corina webservice is written entirely in PHP.  Like the Desktop Client, the server is developed with Eclipse so most of the setup steps are identical (see chapter \ref{txt:devDesktop}).  You will, however, probably want to install the PHP development plugin so that you get syntax highlighting etc.  See the Eclipse PDT website (\url{http://www.eclipse.org/pdt/}) for further information.


\section{Server package}
\label{txt:serverPackage}
\index{Packaging!Server}
The Ubuntu server package can be built using the server-package Ant target described in section \ref{txt:buildScript}.  This script is considerably more complicated than the desktop packager so is worth describing further here in case alterations are required.

The script makes use of Ubuntu's excellent package management system to handle the dependencies.  Adding or editing dependencies is simply a matter of changing the `depends' attribute in the build.xml file.  

The database files are installed to `/usr/share/corina-server' and the webservices files to `/var/www/corina-webservice'.  The bulk of the complicated work is done by the script specified in the postinst attribute in the build.xml file (i.e.\ native/LinBuild/corina-server).  This is an interactive PHP command line script that can also be run by users at a later date.  After the server package has been installed this script is automatically run and asks the users questions to assist with setting up the server.  The configuration steps are:

\begin{itemize*}
 \item Generated scripts from templates
 \item Configure PostgreSQL database, creating users and/or database if requested otherwise obtaining details if they already exist
 \item Configure PostgreSQL to allow access to the specified database user
 \item Configure Apache to access the webservice
 \item Verify setup by checking Apache and PostgreSQL are running, that the webservice is accessible, the database is accessible and that various configuration files can be read
\end{itemize*}

This configuration script is therefore where the majority of alterations are made to the way the server is packaged.  

\index{Developing|)}