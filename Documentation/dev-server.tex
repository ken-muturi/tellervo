\chapter{Developing Corina Server}
\index{Developing|(}
\index{Developing!Webservice}
\index{Webservice!Developing}

The Corina server is made up of a PHP webservice run by Apache, connecting to a PostGreSQL database.  

The Corina webservice is written entirely in PHP.  Like the Desktop Client, the server is developed with Eclipse so most of the setup steps are identical (see chapter \ref{txt:devDesktop}).  You will, however, probably want to install the PHP development plugin so that you get syntax highlighting etc.  See the Eclipse PDT website (\url{http://www.eclipse.org/pdt/}) for further information.


\section{Webservice }

\section{Server package}
\label{txt:serverPackage}
\index{Packaging!Server}
The Ubuntu server package is built by Maven at the same time as the desktop package (see section \ref{txt:buildScript}) during the package goal.  

The server packaging is done as a secondary execution of the JDeb plugin.  JDeb is configured in the pom.xml by including all the files that need to be copied along with where in the target file system they should be placed. The database files are installed to `/usr/share/corina-server' and the webservices files to `/var/www/corina-webservice'. 

The metadata for the deb file is included in the control file located in Native/BuildResources/LinBuild/ServerControl.  JDeb makes use of Ubuntu's excellent package management system to handle the dependencies.  Adding or editing dependencies is simply a matter of changing the `depends' attribute control file.  The ServerControl folder also contains a script called postinst which is launched after the installation is complete.  This is used to trigger the interactive script that helps the user configure the Corina server (described further in section \ref{txt:corina-server-script}.  The steps are as follows:  

\begin{itemize*}
 \item Check the user running the script is root as we're doing privileged functions
 \item Generated scripts from templates
 \item Configure PostgreSQL database, creating users and/or database if requested otherwise obtaining details if they already exist
 \item Configure PostgreSQL to allow access to the specified database user
 \item Configure Apache to access the webservice
 \item Verify setup by checking Apache and PostgreSQL are running, that the webservice is accessible, the database is accessible and that various configuration files can be read
 \item Print test report to screen
\end{itemize*}

\subsection{Corina server script}
\label{txt:corina-server-script}

At the heart of most of the configuration and control of the Corina server is the corina-server script.  This is a command line PHP script that is launched after installation and can be re-run by the user to make changes to the configuration.

\section{Making a new release}

As mentioned in section \ref{txt:serverPackage}, the server package is created at the same time as the desktop binaries as part of the Maven package procedure.  There are, however, a number of steps you need to undertake to make sure this goes 



\index{Developing|)}