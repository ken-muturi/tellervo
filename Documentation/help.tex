\chapter{Help and support}


\section{Getting help}

At the moment your options for getting help are largely limited to contacting Peter Brewer!  Once the user-base of Tellervo expands we will set up forums and mailing lists to assist.


\section{Support for future development}
Both Tellervo Desktop and Server are free software available under the General Public License v3 (see appendix \ref{txt:licenseStart}).  This means you are free to use Tellervo in both academic and commercial environments.  However, when we talk about `free software' (as the license explains) we are talking about freedom of use, not free as in price.  Tellervo has inevitably cost a great deal to develop over the years and while you are not asking for a direct contribution, we do need your support for future development.

If there is particular functionality that you would like to see implemented in Tellervo, under the open-source model this can be done in a number of ways:

\begin{description}
 \item[Implement the feature yourself!] -- If you are able to program in Java then we would be delighted to assist you to implement new features.  You could do this in isolation\footnote{Note that although the GPL license allows you do develop Tellervo separately, it does include clauses that require you to make the source code of the software you create also freely available under GPL or a compatible license. If you `fork' the code in this way you will find it increasingly difficult to benefit from improvements made to the official Tellervo code.} but we hope you will do this collaboratively with us and make the new feature available to the rest of our community.  Please contact the developers and we will organize a developers SVN account for you to access and contribute to the source code.
 \item[Request a feature from the developers] -- Contact the developers at Cornell and discuss the feature that you would like implemented.  If the feature is relatively easy to implement and/or deemed useful for the Cornell laboratory then we may be able to implement the feature for you.
 \item[Pay a third party developer] -- If you know a third party developer that can make the changes for you then this is also possible.  Again, we would ask that you do this in consultation with the existing developers so that any improvements can be contributed back to the community.
 \item[Collaborative development] -- If you have an idea for exciting new functionality we would be pleased to discuss the possibility of collaborative development--for example as part of a grant funded project.  The chances of success when applying for infra-structure projects from federal agencies are much greater when proposed as part of a collaborative multi-laboratory project.
\end{description}


